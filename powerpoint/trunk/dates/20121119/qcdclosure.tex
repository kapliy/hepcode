\section{ QCD closure }

\slide{ QCD closure}
{
\iteb
\item Small study looking at single-differential cross-section
\item All plots at detector level
\item Signal and EWK are Powheg+Pythia, tau is Alpgen+Herwig
\item QCD template is one of:
\iteb
\item Heavy-flavor MC
\item Data (anti-isolated control region)
\item Heavy-flavor MC (anti-isolated control region)
\itee 
\itee 

Note the changes in the shape as a function of $\eta$.
}


\slide{ Anti-isolation $Iso^{cone20}/p_{T} > 0.1$ }
{
\centering

\colb[T]
\column{.5\textwidth}
\centering
Muon $\Wminus$ $|\eta|$ \\
\includegraphics[width=1.0\textwidth]<1>{dates/20121119/figures/qcdclosure/isofail0_NEG.pdf}
\includegraphics[width=1.0\textwidth]<2>{dates/20121119/figures/qcdclosure/isofail4_NEG.pdf}
\includegraphics[width=1.0\textwidth]<3>{dates/20121119/figures/qcdclosure/isofail10_NEG.pdf}

\column{.5\textwidth}
\centering
Muon $\Wplus$ $|\eta|$ \\
\includegraphics[width=1.0\textwidth]<1>{dates/20121119/figures/qcdclosure/isofail0_POS.pdf}
\includegraphics[width=1.0\textwidth]<2>{dates/20121119/figures/qcdclosure/isofail4_POS.pdf}
\includegraphics[width=1.0\textwidth]<3>{dates/20121119/figures/qcdclosure/isofail10_POS.pdf}
\cole

\only<1>{\footnotesize{ QCD = heavy-flavor MC  }}
\only<2>{\footnotesize{ QCD = data anti-iso control region. Note \red{dramatic change in shape} }}
\only<3>{\footnotesize{ QCD = MC, but in anti-iso control region. Now it looks quite similar to data }}

}

\slide{ Anti-isolation $Iso^{cone20}/p_{T} = [0.1 - 0.2]$ }
{
\centering

\colb[T]
\column{.5\textwidth}
\centering
Muon $\Wminus$ $|\eta|$ \\
\includegraphics[width=1.0\textwidth]<1>{dates/20121119/figures/qcdclosure/isowind0_NEG.pdf}
\includegraphics[width=1.0\textwidth]<2>{dates/20121119/figures/qcdclosure/isowind4_NEG.pdf}
\includegraphics[width=1.0\textwidth]<3>{dates/20121119/figures/qcdclosure/isowind10_NEG.pdf}

\column{.5\textwidth}
\centering
Muon $\Wplus$ $|\eta|$ \\
\includegraphics[width=1.0\textwidth]<1>{dates/20121119/figures/qcdclosure/isowind0_POS.pdf}
\includegraphics[width=1.0\textwidth]<2>{dates/20121119/figures/qcdclosure/isowind4_POS.pdf}
\includegraphics[width=1.0\textwidth]<3>{dates/20121119/figures/qcdclosure/isowind10_POS.pdf}
\cole

\only<1>{\footnotesize{ QCD = heavy-flavor MC  }}
\only<2>{\footnotesize{ QCD = data anti-iso control region. Note \red{dramatic change in shape} }}
\only<3>{\footnotesize{ QCD = MC, but in anti-iso control region. Now it looks quite similar to data }}

}

\slide{ Observations on QCD closure}
{
\iteb
\item Anti-isolation control region appears to bias $\eta$ shape on the order of 2-3 \%
\item The bias is more pronounced in inverted isolation case (vs a window cut)
\item Never noticed these effects because ratio pad scale used to be set to $\pm 50\%$
\item TODO: check what happens when QCD is normalized in each $\eta$ bin
\itee 

}



\slide{ Aside: TFractionFitter and bin-by-bin variations }
{

\centering
\only<1>{\footnotesize{ Templates returned from TFractionFitter  }}
\only<2>{\footnotesize{ Original templates, but normalization from TFractionFitter }}

\includegraphics[width=0.6\textwidth]<1>{dates/20121119/figures/qcdclosure/tff_mod.png}
\includegraphics[width=0.6\textwidth]<2>{dates/20121119/figures/qcdclosure/tff_orig.png}

TFractionFitter is free to change contents of each bin independently. \\
This means the template shapes may change, and agreement looks better.\\
Of course, if a bin changes too much, the penalty to $\chi^{2}$ gets large.

}
